
\documentclass[a4paper]{memoir}
% \documentclass{book}
\usepackage{coffeenode-tides}


% % http://tex.stackexchange.com/questions/29537/table-can-i-shift-a-column-by-half-the-height-of-a-row?rq=1
% % must use `\multicolumn{1}{l}{...}` in header to avoid shifting there
% \usepackage{collcell}
% \newcommand*{\movedown}[1]{%
%   \smash{\raisebox{-1.3ex}{#1}}}
% \newcolumntype{q}{>{\collectcell\movedown}r<{\endcollectcell}}
%    % (for "quotient")

\usepackage[absolute]{textpos}
\setlength{\TPHorizModule}{1mm}
\setlength{\TPVertModule}{1mm}
% \textblockorigin{hpos}{vpos} % upper left corner of main printing area

% You may give a optional argument to the \Lenv{textblock}
% environment, specifying which point in the box
% is to be placed at the specified point:
% \begin{quote}
% \begin{raggedright}
% \cmd|\begin{textblock}{<hsize>}[<ho>,<vo>](<hpos>,<vpos>)|\\
% text...\\
% |\end{textblock}|
% \end{raggedright}
% \end{quote}
% The coordinates \meta{ho} and \meta{vo} are fractions of the
% width and height of the text box, respectively, and state that the
% box is to be placed so that the reference
% point (\meta{ho},\meta{vo}) within the box is to be placed at the point
% (\meta{hpos},\meta{vpos}) on the page.  The default specification is
% [0,0], the top left of the box: [0,1] would be the bottom left, and
% [0.5,0.5] the middle.



\begin{document}


% \multicolumn{1}{r}{L}

% \begin{textblock}{10}[0.5,0.5](100,100)
\begin{textblock*}{10mm}(100mm,100mm)
helo world!
\end{textblock*}

\begin{textblock}{100}[1,0.5](100,120)
\flushright to the left
\end{textblock}

\begin{textblock}{100}[0,0.5](100,120)
% to the right
\framebox[1.1\width]{Guess I'm framed now!}
\end{textblock}


\begin{tabular}
{ r r l r r r | c | c | c }
\multicolumn{3}{l}{{\color{DarkRed}\scFont{}\large{} Jan} 2014} & H & L & \\

\hline

\cline{2-4}\cline{5-5}\newmoon & {\itFont{} 1}. & {\itFont{}za} &  &  2 : 16 & \\
 &  &  &  8 : 26 & 14 : 41 & \\
\cline{5-5} &  &  & 20 : 46 &  3 : 13 & \\
\cline{2-4} & {\itFont{} 2}. & {\color{DarkRed}\itFont{}zo} &  9 : 15 & 15 : 31 & \\
\cline{5-5} &  &  & 21 : 36 &  4 : 02 & \\

\hline

\multirow{2}{*}{Multirow} & \multirow{2}{*}{the ant knows better} & 9 : 15 \\
C & \multirow{2}{*}{Multirow} & 15 : 31 \\

\\

\hline


\end{tabular}

% \end{foo}
% \end{multicols}
\end{document}

